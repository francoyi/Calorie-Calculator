\documentclass{beamer}
\usetheme{Madrid}
\usecolortheme{default}
\usepackage{graphicx}
\usepackage{listings}
\usepackage{xcolor}
\usepackage{booktabs}
\definecolor{codegreen}{rgb}{0,0.6,0}
\definecolor{codegray}{rgb}{0.5,0.5,0.5}
\definecolor{codepurple}{rgb}{0.58,0,0.82}
\definecolor{backcolour}{rgb}{0.95,0.95,0.92}
\lstdefinestyle{mystyle}{
    backgroundcolor=\color{backcolour},
    commentstyle=\color{codegreen},
    keywordstyle=\color{magenta},
    numberstyle=\tiny\color{codegray},
    stringstyle=\color{codepurple},
    basicstyle=\ttfamily\tiny,
    breakatwhitespace=false,
    breaklines=true,
    captionpos=b,
    keepspaces=true,
    numbers=left,
    numbersep=5pt,
    showspaces=false,
    showstringspaces=false,
    showtabs=false,
    tabsize=2
}
\lstset{style=mystyle}

\title
{CSC207 Final Project: Calorie Calculator}

\subtitle{Group \#14}

\author[Group \#14]{
    Zhengyu Yi \\
    Max (Haotian) Xu \\
    Haoying Zhu \\
    Shourya Harsh Vardhan \\
    Yan Lam \\
    Rasyid Rafi Pamuji
}
\institute{University of Toronto}

\date{\today}


% --- UofT Blue Theme Settings ---
\definecolor{uoftblue}{RGB}{6,41,88}
\usecolortheme[named=uoftblue]{structure}

\setbeamercolor{palette primary}{bg=uoftblue,fg=white}
\setbeamercolor{palette secondary}{bg=uoftblue!80!white,fg=white}
\setbeamercolor{palette tertiary}{bg=uoftblue!60!white,fg=white}

\begin{document}

\frame{\titlepage}

% \begin{frame}
% \frametitle{Table of Contents}
% \tableofcontents
% \end{frame}

% =========================================
% SECTION 1: OVERVIEW
% =========================================
\section{Project Overview}

% --- Slide 1: What is CalorieCalc? ---
\begin{frame}
    \frametitle{Project Overview: CalorieCalc}
    \begin{columns}
        \column{0.5\textwidth}
        \Large
        \textbf{What is it?}
        \vspace{0.3cm}

        \normalsize
        A smart nutrition tracking application built with Java Swing.

        \vspace{0.8cm}

        \Large
        \textbf{Our Goal}
        \vspace{0.3cm}

        \normalsize
        To help users manage daily intake through API integration and intelligent algorithms.

        \column{0.5\textwidth}
        \centering
        \includegraphics[width=0.95\textwidth]{example-image}
        \\ \tiny{Figure: Main Application Interface}
    \end{columns}
\end{frame}

% --- Slide 2: MVP Key Features ---
\begin{frame}
    \frametitle{Key Features (MVP)}
    \begin{itemize}
        \setlength\itemsep{1em}

        \item \textbf{Smart Search}:
        OpenFoodFacts API integration for real-time data.

        \item \textbf{Personalized Tracking}:
        Custom TDEE/BMR goals tailored to user metrics.

        \item \textbf{Intelligent Recommendations}:
        DP-based meal suggestions within calorie budget.

        \item \textbf{Secure Persistence}:
        Local JSON storage with atomic write safety.

        \item \textbf{Data Visualization}:
        Nutritional breakdown charts.
    \end{itemize}
\end{frame}

\begin{frame}
    \frametitle{Technical Highlights}

    \begin{block}{External API: OpenFoodFacts}
        \begin{itemize}
            \item Real-time nutritional data fetching.
            \item \textbf{Optimization}: Implemented \textbf{In-Memory Caching} to handle rate limits.
        \end{itemize}
    \end{block}

    \vspace{0.5cm}

    \begin{alertblock}{Data Persistence}
        \begin{itemize}
            \item \textbf{Format}: JSON (via Jackson library).
            \item \textbf{Safety}: Uses \textbf{Atomic File Writes} (\texttt{Files.move}) to ensure integrity.
        \end{itemize}
    \end{alertblock}

\end{frame}

% =========================================
% SECTION 2: INDIVIDUAL WALKTHROUGHS
% =========================================
\section{Individual Walkthroughs}

% --- Member 1: Zhengyu Yi (API) ---
\begin{frame}[fragile]
    \frametitle{Use Case: Search \& Add API Food (Zhengyu Yi)}
    \textbf{User Story:} "Search for food via API and add it to today's log."
    \begin{columns}
        \column{0.5\textwidth}
        \begin{itemize}
            \item \textbf{Key Tech}: \texttt{OpenFoodFactsClient} with Caching.
            \item \textbf{Logic}: Handles JSON parsing and mapping to \texttt{FoodItem}.
            \item \textbf{Feature}: Supports natural language parsing (e.g., "100g chicken").
        \end{itemize}
        \column{0.5\textwidth}
        \centering \textbf{[TODO: Insert API Search Screenshot]}
        % \includegraphics[width=0.9\textwidth]{ui_api.png}
    \end{columns}
\end{frame}

% --- Member 2: Janice Lam (Local Food) ---
\begin{frame}[fragile]
    \frametitle{Use Case: Create Local Food (Janice Lam)}
    \textbf{User Story:} "Manually create custom foods when API data is missing."
    \begin{columns}
        \column{0.5\textwidth}
        \begin{itemize}
            \item \textbf{UI}: Dynamic \texttt{JTable} to add/remove ingredients.
            \item \textbf{Polymorphism}: Implements \texttt{LocalFoodItem} which shares the \texttt{FoodItem} interface.
        \end{itemize}
        \column{0.5\textwidth}
        \centering \textbf{[TODO: Insert Create Food Dialog Screenshot]}
        % \includegraphics[width=0.9\textwidth]{ui_create.png}
    \end{columns}
\end{frame}

% --- Member 3: Haoying Zhu (Goals/History) ---
\begin{frame}[fragile]
    \frametitle{Use Case: Goal Setting \& History (Haoying Zhu)}
    \textbf{User Story:} "Set daily calorie goals and view past consumption."
    \begin{columns}
        \column{0.5\textwidth}
        \begin{itemize}
            \item \textbf{Persistence}: Manages \texttt{UserSettings} retrieval.
            \item \textbf{UI Logic}: Visual feedback (Red text) when intake $>$ goal.
            \item \textbf{Navigation}: Browse history by date.
        \end{itemize}
        \column{0.5\textwidth}
        \centering \textbf{[TODO: Insert Main Panel Screenshot]}
        % \includegraphics[width=0.9\textwidth]{ui_history.png}
    \end{columns}
\end{frame}

% --- Member 4: Max Xu (Recommendation) ---
\begin{frame}[fragile]
    \frametitle{Use Case: Meal Recommendations (Max Xu)}
    \textbf{User Story:} "Get food recommendations based on remaining calorie budget."
    \begin{columns}
        \column{0.5\textwidth}
        \begin{itemize}
            \item \textbf{Algorithm}: \textbf{Dynamic Programming} (Knapsack-style).
            \item \textbf{Complexity}: Optimizes food combinations to maximize utility under the calorie cap.
        \end{itemize}
        \column{0.5\textwidth}
        \centering \textbf{[TODO: Insert Recommendation Screenshot]}
        % \includegraphics[width=0.9\textwidth]{ui_recommend.png}
    \end{columns}
\end{frame}

% --- Member 5: Shourya Harsh Vardhan (TDEE) ---
\begin{frame}[fragile]
    \frametitle{Use Case: Calculate BMR/TDEE (Shourya Harsh Vardhan)}
    \textbf{User Story:} "Calculate Basal Metabolic Rate to understand energy needs."
    \begin{columns}
        \column{0.5\textwidth}
        \begin{itemize}
            \item \textbf{Design Pattern}: \textbf{Strategy Pattern} (\texttt{BMRFormula}).
            \item \textbf{Clean Arch}: Pure Interactor logic with strict Input/Output Boundaries.
        \end{itemize}
        \column{0.5\textwidth}
        \centering \textbf{[TODO: Insert TDEE Screenshot]}
        % \includegraphics[width=0.9\textwidth]{ui_tdee.png}
    \end{columns}
\end{frame}

% --- Member 6: Rasyid Rafi Pamuji (Visualization) ---
\begin{frame}[fragile]
    \frametitle{Use Case: Visualize Nutrition (Rasyid Rafi Pamuji)}
    \textbf{User Story:} "Visualize daily protein, carb, and fat distribution."
    \begin{columns}
        \column{0.5\textwidth}
        \begin{itemize}
            \item \textbf{Goal}: Provide visual insights into diet balance (Macronutrients).
            \item \textbf{Implementation}: Processes \texttt{NutritionValues} from Daily Logs.
        \end{itemize}
        \column{0.5\textwidth}
        \centering \textbf{[TODO: Insert Chart Screenshot]}
        % \includegraphics[width=0.9\textwidth]{ui_chart.png}
    \end{columns}
\end{frame}

% =========================================
% SECTION 3: ARCHITECTURE & DESIGN
% =========================================
\section{Architecture \& Design}

\begin{frame}
    \frametitle{Clean Architecture & SOLID}
    \begin{columns}
        \column{0.55\textwidth}
        \begin{itemize}
            \item \textbf{Dependency Inversion (DIP)}:
            \begin{itemize}
                \item High-level Interactors depend on \textbf{Interfaces} (OutputBoundary, Repository), never on concrete implementations.
            \end{itemize}
            \vspace{0.3cm}
            \item \textbf{Single Responsibility (SRP)}:
            \begin{itemize}
                \item Strict separation: \texttt{UnitParser} (Parsing) vs \texttt{FoodLogService} (Logic) vs \texttt{JsonRepository} (IO).
            \end{itemize}
        \end{itemize}

        \column{0.45\textwidth}
        \centering
        % TODO:replace w ca_diagram.png
        \includegraphics[width=\textwidth]{example-image}
        \\ \tiny{Project Dependency Graph}
    \end{columns}
\end{frame}

\begin{frame}
    \frametitle{Design Patterns}
    \begin{enumerate}
        \item \textbf{Strategy Pattern} (in TDEE):
        \begin{itemize}
            \item \texttt{BMRFormula} interface allows hot-swapping calculation algorithms (e.g., Mifflin-St Jeor).
        \end{itemize}
        \item \textbf{Template Method Pattern} (in Persistence):
        \begin{itemize}
            \item \texttt{AbstractJsonRepository} defines the standard atomic write skeleton, subclasses define types.
        \end{itemize}
        \item \textbf{Facade Pattern} (in Service):
        \begin{itemize}
            \item \texttt{FoodLogService} simplifies complex interactions between Repositories and API Clients for the UI.
        \end{itemize}
    \end{enumerate}
\end{frame}

% =========================================
% SECTION 4: QUALITY & PROCESS
% =========================================
\section{Quality \& Process}

\begin{frame}
    \frametitle{Code Quality \& Process}
    \small
    \begin{itemize}
        \item \textbf{Testing Strategy}:
        \begin{itemize}
            \item Achieved \textbf{100\% Coverage} on critical Interactors (TDEE, MealRecommender).
            \item Integration tests for OpenFoodFacts API.
        \end{itemize}
        \item \textbf{Collaboration}:
        \begin{itemize}
            \item Feature branches workflow.
            \item Pull Requests with meaningful code reviews.
        \end{itemize}
    \end{itemize}
    \vspace{0.3cm}
    \centering
    % TODO:Replace with the green bar of test coverage or the screenshot of the Git PR.
    \includegraphics[width=0.4\textwidth]{example-image}
    \\ \tiny{Evidence of Test Coverage \& Git Flow}
\end{frame}

\begin{frame}
    \centering
    \Huge Thank You! \\
    \vspace{1cm}
    \Large Questions?
\end{frame}

% =========================================
% APPENDIX: Q&A DEFENSE
% =========================================
\appendix
\begin{frame}
    \frametitle{Appendix: Architecture Decision}
    \textbf{Q: Why does the View talk to Service directly in some cases?}
    \vspace{0.5cm}
    \begin{itemize}
        \item Ideally, we would use a Controller for every single interaction.
        \item However, for complex Swing components requiring background threading (like the dynamic Food Table), we used the \texttt{FoodLogService} as a \textbf{Facade}.
        \item This keeps the code pragmatic while still ensuring Business Logic is decoupled from Data Sources via Repositories.
    \end{itemize}
\end{frame}

\end{document}
