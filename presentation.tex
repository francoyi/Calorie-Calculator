\documentclass{beamer}
\usetheme{Madrid}
\usecolortheme{default}

\title
{CSC207 Final Project: Calorie Calculator}

\subtitle{Group \#14}

\author[Yi, Xu, Zhu, Vardhan, Lam, Pamuji] % (optional, for multiple authors)
{}

\institute % (optional)
{

  \inst{1}%
    Zhengyu Yi
  \and
  \inst{2}%
    Max (Haotian) Xu
  \and
  \inst{3}%
    Haoying Zhu
  \and
  \inst{4}%
    Shourya Harsh Vardhan
  \and
  \inst{5}%
    Yan Lam
  \and
  \inst{6}%
    Rasyid Rafi Pamuji
}

\date{\today}


\definecolor{uoftblue}{RGB}{6,41,88}
\setbeamercolor{titlelike}{bg=uoftblue}
\setbeamerfont{title}{series=\bfseries}

\begin{document}

\frame{\titlepage}

% \begin{frame}
% \frametitle{Table of Contents}
% \tableofcontents
% \end{frame}

\begin{frame}
    \frametitle{User Stories (i.e., what functionality is in your team's MVP?)}
    \begin{itemize}
    \item Explain what functionality your program provides.
    \item You can use whatever format you want — just make sure to clearly and concisely convey this information.
    \item Advice: avoid too many words here and don’t plan to read a full paragraph of text on the slide when presenting — just highlight the key points.
    \item Focus on the WHAT and not the HOW!
    \item Recommended time limit: [1 minute]
    \end{itemize}
\end{frame}

\begin{frame}
    \frametitle{API Usage}
    \begin{itemize}
        \item What API(s) did your team use?
        \item Make sure to show appropriate information (briefly).
        \item Recommended time limit: [30 seconds]
    \end{itemize}
\end{frame}

\begin{frame}
    \frametitle{Data Persistence}
    \begin{itemize}
        \item What data is persistent in your program?
        \item Either here or in one of the use case demonstrations of functionality, data persistence should be demonstrated.
        \item Recommended time limit: [30 seconds]
    \end{itemize}
\end{frame}

\begin{frame}
    \frametitle{TEMPLATE: Use Case Walkthrough (1 minute)}
    \begin{itemize}
        \item Briefly state your user story and which associated use case you will focus on
        \item Show the before and after views for when the use case executes
        \item Show a UML class diagram for the use case; it should be clear from the diagram that your code adheres to CA!
        \item Show the code for your Use Case Interactor class.
        \item Discuss the flow of control when your use case executes.
        \item \ [This should be rehearsed so that it is around 1 minute per member]
    \end{itemize}
\end{frame}

\begin{frame}
    \frametitle{Zhengyu Yi Use Case Walkthrough (1 minute)}
    \begin{itemize}
        \item Briefly state your user story and which associated use case you will focus on
        \item Show the before and after views for when the use case executes
        \item Show a UML class diagram for the use case; it should be clear from the diagram that your code adheres to CA!
        \item Show the code for your Use Case Interactor class.
        \item Discuss the flow of control when your use case executes.
        \item \ [This should be rehearsed so that it is around 1 minute per member]
    \end{itemize}
\end{frame}

\begin{frame}
    \frametitle{Max Xu Use Case Walkthrough (1 minute)}
    \begin{itemize}
        \item Briefly state your user story and which associated use case you will focus on
        \item Show the before and after views for when the use case executes
        \item Show a UML class diagram for the use case; it should be clear from the diagram that your code adheres to CA!
        \item Show the code for your Use Case Interactor class.
        \item Discuss the flow of control when your use case executes.
        \item \ [This should be rehearsed so that it is around 1 minute per member]
    \end{itemize}
\end{frame}

\begin{frame}
    \frametitle{Haoying Zhu Use Case Walkthrough (1 minute)}
    \begin{itemize}
        \item Briefly state your user story and which associated use case you will focus on
        \item Show the before and after views for when the use case executes
        \item Show a UML class diagram for the use case; it should be clear from the diagram that your code adheres to CA!
        \item Show the code for your Use Case Interactor class.
        \item Discuss the flow of control when your use case executes.
        \item \ [This should be rehearsed so that it is around 1 minute per member]
    \end{itemize}
\end{frame}

\begin{frame}
    \frametitle{Shourya Harsh Vardhan Use Case Walkthrough (1 minute)}
    \begin{itemize}
        \item Briefly state your user story and which associated use case you will focus on
        \item Show the before and after views for when the use case executes
        \item Show a UML class diagram for the use case; it should be clear from the diagram that your code adheres to CA!
        \item Show the code for your Use Case Interactor class.
        \item Discuss the flow of control when your use case executes.
        \item \ [This should be rehearsed so that it is around 1 minute per member]
    \end{itemize}
\end{frame}

\begin{frame}
    \frametitle{Yan Lam Use Case Walkthrough (1 minute)}
    \begin{itemize}
        \item Briefly state your user story and which associated use case you will focus on
        \item Show the before and after views for when the use case executes
        \item Show a UML class diagram for the use case; it should be clear from the diagram that your code adheres to CA!
        \item Show the code for your Use Case Interactor class.
        \item Discuss the flow of control when your use case executes.
        \item \ [This should be rehearsed so that it is around 1 minute per member]
    \end{itemize}
\end{frame}

\begin{frame}
    \frametitle{Rasyid Rafi Pamuji Use Case Walkthrough (1 minute)}
    \begin{itemize}
        \item Briefly state your user story and which associated use case you will focus on
        \item Show the before and after views for when the use case executes
        \item Show a UML class diagram for the use case; it should be clear from the diagram that your code adheres to CA!
        \item Show the code for your Use Case Interactor class.
        \item Discuss the flow of control when your use case executes.
        \item \ [This should be rehearsed so that it is around 1 minute per member]
    \end{itemize}
\end{frame}

\begin{frame}
    \frametitle{Design (Recommended time: 2.5 minutes)}
    \begin{itemize}
        \item How does your program adhere to SOLID?
        \begin{itemize}
            \item Should aim to talk about two specific examples present in your project; touching on at least two principles.
        \end{itemize}
        \item How does your program adhere to the Clean Architecture?
        \begin{itemize}
            \item This can be very short, or most likely skipped entirely, depending on what each member talked about in the previous use case part.
        \end{itemize}
        \item What is the best example of a design pattern used in your program?
        \begin{itemize}
            \item Aim to talk about one design pattern that your team introduced — don't talk about a pattern that was already implemented in the starter code to earn full marks here.
        \end{itemize}
        \item As appropriate, your team should make use of diagrams and other visuals to convey this information.
        \begin{itemize}
            \item Almost always, diagrams will be more effective than showing the details of the code! 

[Image of relevant design diagram]

        \end{itemize}
        \item Roughly, your team might aim for something 2–4 slides in total about design, but the exact number can vary depending on how you are presenting the information.
    \end{itemize}
\end{frame}

\begin{frame}
    \frametitle{Functionality Demonstration (Recommended time: 3.5 minutes)}
    \begin{itemize}
        \item Make sure to time things out so that your team can demonstrate at least the core functionality of your MVP.
        \item Feel free to demo live or show recordings, but you should have a working version of the program available to demo specific functionality live immediately after the 15-minute presentation if there are questions.
        \item Make sure to especially rehearse the demo, as it can be easy to spend too much time here!
        \item Focus on the most interesting parts of the program (e.g., don't spend time on things like signing up a user, changing password, or similar unless it is central to the program)
        \item NOTE: depending on how your team decides to present, it might be appropriate to wait until the very end to do your demo.
    \end{itemize}
\end{frame}

\begin{frame}
    \frametitle{Code Organization (Recommended time: 30 seconds)}
    \begin{itemize}
        \item Talk about how your team organized your code.
        \item How did you package your code?
        \item Tip: make sure you follow proper naming conventions throughout your code, especially package names.
    \end{itemize}
\end{frame}

\begin{frame}
    \frametitle{Code Quality (Recommended time: 30 seconds)}
    \begin{itemize}
        \item Briefly explain how your team was able to maintain code quality.
        \begin{itemize}
            \item You might discuss how you used Checkstyle or similar tools.
            \item You might discuss your approach to pull requests and code reviews.
            \item For the exceptional level, reminder that your team needs to explicitly draw attention to a representative pull request demonstrating your team's approach to code quality.
        \end{itemize}
    \end{itemize}
\end{frame}

\begin{frame}
    \frametitle{The End}
    \begin{itemize}
        \item Time permitting, your team can highlight anything else about the project that you want to share.
        \item You’ll likely fill the time with covering the required rubric elements, so only say more if you have extra time to fill.
        \item The recommended time limits would put your presentation around 15 minutes, so your team most likely won't have time for anything else.
    \end{itemize}
\end{frame}

\begin{frame}
\frametitle{Highlighting text}

In this slide, some important text will be
\alert{highlighted} because it's important.
Please, don't abuse it.

\begin{block}{Remark}
Sample text
\end{block}

\begin{alertblock}{Important theorem}
Sample text in red box
\end{alertblock}

\begin{examples}
Sample text in green box. The title of the block is ``Examples".
\end{examples}
\end{frame}

\end{document}

About

    About us
